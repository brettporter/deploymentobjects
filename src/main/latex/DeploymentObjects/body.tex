

\chapter{INTRODUCTION}
\label{chap:intro}

\section{Problem Statement}

Just as administering most production environments is a labor-intensive manual
process, there is no such thing as an installer for distributed systems.
Deploying distributed software systems today involves a sequence of steps across
multiple systems which are individually automated but rarely automated as a
group. In instances when deployment is automated, it is accomplished by 1-off
scripts too tightly coupled to external communication to be reusable. So many
orthogonal concerns exist that efforts produce quickly growing balls of mud.

As a result, consider the task of creating a distributed Continuous Integration
(CI) server: many organizations run CI servers on a single machine but a
distributed CI server farm requires extensive setup and configuration for each
company that wishes to run load tests, so only the most well-funded
organizations follow the practice; small projects don't follow the
practice. Production environments are deployed using a different process from
that which occurs from the CI server. Certain steps must be done manually,
leaving the possibility for user error. The result is that the devevelopment
environment differs from the production environment, resulting in a higher
frequency of errors when software is released, causing various forms of pain to
the organization.

\section{Background}

\subsection{Existing Methodologies}
\subsubsection{Manual}
\label{existing:manual}

Consider a site with seperate development and operations groups.  The
development group produces a release, and the operations group follows the
installation documentation and manually installs the application in the
production environment.

\subsubsection{Systems Configuration}
\label{existing:sysconfig}

Consider a site early in their attempts to automate their environment.
Developers code a web application, run their builds, and copy the artifacts to a
test server.  On the test server managed by an automated configuration tool, the
developers restart a container and either tail a logfile until it says it's
ready, or waits the usual amount of time until they think the application has
finished deploying. Then they open up a browser, enter the URL, and look to
see if the page loads.
 
Some weeks go by, QA says the release appears ready to go to
production. Unfortunately, nobody ran the updated configuration management
config in QA and your updates take down production, so they have to back out
your changes. 


\subsubsection{A Continuous Delivery Pipeline}
\label{existing:continuousdeliverpipeline}

In a continuous delivery \cite{CONTINUOUSDELIVERY} pipeline, version
control systems are used, artifacts are stored in artifact repositories, systems
are configured using configuration management tools, and deployment scripts are
written using orchestration tools.

Etsy has such a implementation, and they require all new engineers to deploy
to production on their first day on the job \cite{ETSYDEPLOYDAYONE}.


\subsubsection{Limitations of Existing Methodologies}


Although \ref{existing:manual} and \ref{existing:sysconfig} can be automated
by widely-available tools, even in \ref{existing:continuousdeliverpipeline},
there are few standards for integrating the pieces, and the market
for configuration management tools is highly fragmented, so every organization
implements the pipeline differently.


\subsection{History}

\subsubsection{Shell Scripts}

Automated system configuration and deployment tools have existed in various
forms for years.  The most basic such tool is written in a shell scripting
language such as the Bourne shell, created in 1977 \cite{BOURNESHELL}, and
solves a single purpose, such as starting a network interface on a host.  It's successor,
Bourne-Again Shell (bash) scripts combine to control the majority of start up
tasks for modern System V and Berkeley UNIX variants and is extensively used in
industry.  The C shell (csh) and its descendants dominate systems programming in
academic environments.  


\subsubsection{Provisioning}

Jumpstart was developed for Solaris in 1994  \cite{JUMPSTART} which reads
installation options from a file when provisioning a Solaris host, freeing up
the administrator from repeatedly selecting the same options from the user
interface manually.  Likewise, Linux vendors developed analogous tools (such as
Kickstart, packaged by Red Hat) starting with early version of their
distributions, and the ability quickly became widespread.

\subsubsection{Configuration Management}

Configuration Management (CM) tools for systems began with CFEngine, a
post-doctoral open-source project started by Mark Burgess at Oslo University in
1993 \cite{CFENGINE}.  After their operating systems have been configured such
as with Jumpstart or Kickstart, CFEngine allows a systems administrator to group systems
and configure them in a uniform and operating-system-independent manner using
human-readable configuration files.  Being the first tool of its kind, CFEngine
continues to be developed and is the most widely-used CM tool in use today.  It
is written in C.

Commercial CM tools have also been developed, such as BMC BladeLogic, HP
OpenView, and IBM Tivoli, but due to their high cost, open-source tools dominate
the market.

In 2005, CFEngine contributor and former BMC employee Luke Kanies developed
Puppet \cite{LUKEKANIESINTERVIEW}.  It was written in Ruby and quickly gathered
a large community by being more easily extended than CFEngine.  Administrators use Puppet's declarative
external Domain Specific Language (DSL) \cite{DSL} to create manifests that
define the desired state of a system; daemons then alter the current state of the system so
it becomes the desired state.  For example, managing users or setting up NFS
mounts can be done by using the provided DSL.  The DSL can be extended by
developing modules in Ruby.  The Puppet community has provided a large number of
modules that simplify and standardize many common administration tasks, such as
configuring a Hadoop cluster or a DNS server via Puppet Forge, a site built for
sharing modules.

In 2009, Puppet contributor Jesse Robbins released yet another alternative,
Chef, which uses an internal DSL rather than an external DSL \cite{DSL}; the
administrator develops in pure Ruby instead of Puppet's proprietary language, which is not
turing-complete.

In March-April 2012, Netflix Cloud Architect Adrian Cockroft wrote a series
of controversial \cite{INFOQNOOPS} blog posts describing how DevOps supplanted
the traditional role of Operations by automating systems provisioning and continuous delivery 
\cite{NETFLIXNOOPS}.

\subsubsection{Orchestrated Deployment}

One limitation of CM tools such as Puppet is that when the administrator alters
the configuration that is shared between nodes, the change only occurs when a
timer on the node expires, causing it to check for changes to apply.  The only
way to coordinate changes across nodes is by using an external tool to trigger
the execution.  Until recently, attempts to provide provisioning orchestration
were limited in design to SSH-for-loops.

In 2011, Canonical introduced Juju \cite{JUJU} (formerly named Ensemble), an
orchestration tool for the Ubuntu Linux distribution, which is designed around
topologies of service deployment, but also handles provisioning tasks to cloud
architecture.  The orchestration components that users develop to engage the
orchestration harness are called Charms.  As of this writing, only cloud
architectures are supported, and only Ubuntu.

Following a 2010 discussion \cite{ANSIBLEYAML} with the author of this thesis,
developer Michael DeHaan introduced Ansible, an orchestration platform where
users define tasks called PlayBooks in YAML, which coordinates SSH and
messaging tasks \cite{ANSIBLE}.  Ansible embeds configuration management
functionality and coordinates tasks to services that expose JSON objects.



\chapter{ARCHITECTURE}

\section{Architectural Overview}
Here I introduce a novel DeploymentObjects tool that attempts to reduce the
pains of deploying and maintaining applications by making tasks previously
impractical to automate practical.  The DeploymentObjects project provides a
user-extensible external Domain-Specific Language (DSL) that describes
distributed automation tasks using the ANTLR meta-compiler but also allows the
user to interface with its internal DSL with a reference implementation written
in Java (implementations in Ruby, Perl, and Python will follow).  It introduces
the Deployment Object design pattern and provides a Domain-Driven Design (DDD)
\cite{DDD} \cite{IMPLEMENTINGDDD} for achieving the intended goals, promoting
code reuse and extension while being covered by extensive unit and integration tests.  It does not reinvent
wheels such as perfectly good configuration management tools but provides
sensible hooks with which to trigger them.  It employs multiple methods for
dispatching actions: SSH for bootstrapping and simple tasks, and Zookeeper for
coordinating distributed tasks.  Users can specify their current environments
using a simple YAML state repository or loaded into Zookeeper where they can be
used as a blackboard watched by many nodes.  Finally, to deal with events that
occur such as errors appearing in a logfile, developers can employ grammars to
trigger events, triggering action elsewhere.


\section{Initial User Stories}

To to prove the concept and flesh out the object model, 3 use cases were built.

\subsection{Create an environment}

In this use case, we create CentOs 6 virtual machines using kickstart and add
them to a naming service, initially POSIX hosts files, to create environment to
which one could deploy applications.

Although more time consuming to build and provision, automated bare-metal system
installation is preferable to cloning existing VMs since the process is
transparent and can be updated at a later date.  Cloned VMs from a gold image
suffer when changes must be made to them, such as updating the operating system
to a newer version: the user must rely on a documented process that was followed
and is up to date.

A reasonable middle ground may be followed which does both: one uses a
source-controlled process to create a gold image using an automated process; one
can clone the gold image when new VMs are needed; should a change to the gold
image be required, one changes the source and builds a new gold image that can
be cloned.  Doing both capitalizes on the speed of copying VMs while providing a
repeatable means of creating the gold image.

This is accomplished by writing kickstart files for each node, connecting via
SSH to a hypervisor capable of creating virtual machines, and executing a script
that initiates the creation of a node.

For this implementation, several assumptions are made.

It is assumed the user does not control the DNS server he uses.  This is typical
for developers who submit code releases to systems administrators in an
operations group who controls production servers.  Many operations already have
the process of adding hosts to DNS servers automated, so we don't reinvent the
wheel. If such a group were to use the system, they would just need to implement
an interface that calls their automation framework.  It is assumed that name
resolution through hosts files is sufficient.

It is assumed that the operating system to be GNU/Linux. Other operating systems
will be supported in time.

This is a useful use case because creating a group of VMs for the purpose of
deploying and testing code from a Continuous Integration (CI) server currently
requires some special tooling to overcome several problems.  First, using a
bare-metal installation process, once VMs finish their installation, they shut
down and must be restarted.  Next, their hostnames must be set and their IPs
must be obtained programmatically and saved to some type of naming service.
Once all nodes are provisioned, they must notify the process on the CI server
that provisioning is complete so it can proceed with deploying and testing the
code on those nodes.  Finally, it is also beneficial because one can employ the
same code base to deploy the application to a production environment.

\subsection{Deploy code to the environment}

todo

\subsection{When a significant event is reported in a log file, trigger an
action to occur in that environment}

todo

\section{Ubiquitous Language}

DDD places a high priority on Ubiquitous Language, which strives to get the
terms used by domain experts to be the same as those software developers.  In a
DDD implementation, code design such as class, variable, and method names
closely resemble the technical jargon used by the experts.  An attempt at
defining a ubiquitous language for DeploymentObjects produced the following
language, with key modeling concepts in CAPS:

ENVIRONMENTS include HOSTS optionally organized by ROLES.  ENVIRONMENTS have
identity and they are the aggregate root for HOSTS and ROLES.

A DISTRIBUTION is a mapping of APPLICATION DATA and ARTIFACTS to HOSTS by their
ROLE.  A USER can DEFINE a DISTRIBUTION.

A DEPLOYMENT is built from a DISTRIBUTION applied to an ENVIRONMENT.

HOSTS have access to STORAGE, which may be LOCAL storage on that HOST, an NFS
server, checkpointed DISK IMAGES available to a HYPERVISOR, or ONLINE STORAGE
such as Amazon�s S3. APPLICATION DATA resides on STORAGE.

A HYPERVISOR is a HOST with the ability to CONTROL the POOL of HOSTS running
inside it, performing actions like creating HOSTS using unallocated disk and cpu
resources.  Amazon�s EC2 also provides the ability to CONTROL a POOL of HOSTS,
but unlike a HYPERVISOR, it is not a HOST.

A PROCEDURE is composed of a series of STEPS that can be run locally or
DISPATCHED to another HOST.  A PROCEDURE can be used to CREATE an ENVIRONMENT,
or to change the STATE of an existing ENVIRONMENT.  A USER can DEFINE a
PROCEDURE.

A JOB is what is run at the time a USER requests that a PROCEDURE be applied to
an ENVIRONMENT.  The system can be organized such that when an error takes
place, it triggers some action by RUNNING a JOB.  A running JOB updates the
state of the ENVIRONMENT through EVENTS.

A PACKAGE is an ARTIFACT with an externally-controlled resource STATE, whereas a
BUNDLE is an ARTIFACT without externally-controlled resource STATE.

RPM is a type of PACKAGE, and TAR and JAR are types of BUNDLE.

ARTIFACTS can even be a collection of uncommitted source files in a directory of
the user�s HOST, but it is neither a PACKAGE, nor a BUNDLE.

YUM is an ARTIFACT REPOSITORY that makes RPM PACKAGES available.  A MAVEN
REPOSITORY is an ARTIFACT REPOSITORY that makes JAR BUNDLES available.  A
VERSIONED DIRECTORY

A USER specifies a PROCEDURE contained in the MODEL to run in a specific
ENVIRONMENT.

Some common PROCEDURES include:

an ENVIRONMENT CREATION PROCEDURE a BUILD PROCEDURE gets source code from a
SOURCE REPOSITORY to create an PACKAGE or BUNDLE and loads it into an ARTIFACT
REPOSITORY and an INSTALLATION PROCEDURE which loads an ARTIFACT from a
REPOSITORY onto a HOST by its ROLE.
An ENVIRONMENT built from SNAPSHOT STORAGE can be reverted to it�s original
STATE.

\chapter{RESULTS}

